\documentclass[10pt,letterpaper]{article}\usepackage[]{graphicx}\usepackage[]{color}
%% maxwidth is the original width if it is less than linewidth
%% otherwise use linewidth (to make sure the graphics do not exceed the margin)
\makeatletter
\def\maxwidth{ %
	\ifdim\Gin@nat@width>\linewidth
	\linewidth
	\else
	\Gin@nat@width
	\fi
}
\makeatother

\definecolor{fgcolor}{rgb}{0.345, 0.345, 0.345}
\newcommand{\hlnum}[1]{\textcolor[rgb]{0.686,0.059,0.569}{#1}}%
\newcommand{\hlstr}[1]{\textcolor[rgb]{0.192,0.494,0.8}{#1}}%
\newcommand{\hlcom}[1]{\textcolor[rgb]{0.678,0.584,0.686}{\textit{#1}}}%
\newcommand{\hlopt}[1]{\textcolor[rgb]{0,0,0}{#1}}%
\newcommand{\hlstd}[1]{\textcolor[rgb]{0.345,0.345,0.345}{#1}}%
\newcommand{\hlkwa}[1]{\textcolor[rgb]{0.161,0.373,0.58}{\textbf{#1}}}%
\newcommand{\hlkwb}[1]{\textcolor[rgb]{0.69,0.353,0.396}{#1}}%
\newcommand{\hlkwc}[1]{\textcolor[rgb]{0.333,0.667,0.333}{#1}}%
\newcommand{\hlkwd}[1]{\textcolor[rgb]{0.737,0.353,0.396}{\textbf{#1}}}%
\let\hlipl\hlkwb

\usepackage{framed}
\makeatletter
\newenvironment{kframe}{%
	\def\at@end@of@kframe{}%
	\ifinner\ifhmode%
	\def\at@end@of@kframe{\end{minipage}}%
\begin{minipage}{\columnwidth}%
	\fi\fi%
	\def\FrameCommand##1{\hskip\@totalleftmargin \hskip-\fboxsep
		\colorbox{shadecolor}{##1}\hskip-\fboxsep
		% There is no \\@totalrightmargin, so:
		\hskip-\linewidth \hskip-\@totalleftmargin \hskip\columnwidth}%
	\MakeFramed {\advance\hsize-\width
		\@totalleftmargin\z@ \linewidth\hsize
		\@setminipage}}%
{\par\unskip\endMakeFramed%
	\at@end@of@kframe}
\makeatother

\definecolor{shadecolor}{rgb}{.97, .97, .97}
\definecolor{messagecolor}{rgb}{0, 0, 0}
\definecolor{warningcolor}{rgb}{1, 0, 1}
\definecolor{errorcolor}{rgb}{1, 0, 0}
\newenvironment{knitrout}{}{} % an empty environment to be redefined in TeX

\usepackage{alltt}
\usepackage[top=0.85in,left=1.75in,footskip=0.75in]{geometry}

% amsmath and amssymb packages, useful for mathematical formulas and symbols
\usepackage{amsmath,amssymb}

% Use adjustwidth environment to exceed column width (see example table in text)
\usepackage{changepage}

% Use Unicode characters when possible
\usepackage[utf8x]{inputenc}

% textcomp package and marvosym package for additional characters
\usepackage{textcomp,marvosym}

% cite package, to clean up citations in the main text. Do not remove.
\usepackage{cite}

% Use nameref to cite supporting information files (see Supporting Information section for more info)
\usepackage{nameref,hyperref}

% line numbers
\usepackage[right]{lineno}

% ligatures disabled
\usepackage{microtype}
\DisableLigatures[f]{encoding = *, family = * }

% color can be used to apply background shading to table cells only
\usepackage[table]{xcolor}

% array package and thick rules for tables
\usepackage{array}

% adjust width of tikz tables or figures
\usepackage{adjustbox}

% bold math symbols package
\usepackage{bm}

% nice figures and captions
\usepackage{graphicx}

% diagrams or complicated equations
\usepackage{tikz}

% vertical and horizontal dashed lines
\usepackage{arydshln}

%\usepackage{floatflt}
%\usepackage{nonfloat}
\usepackage{float}
\usepackage{wrapfig}

%\renewcommand{\arraystretch}{1.2}
%\setlength{\tabcolsep}{12pt}

% create "+" rule type for thick vertical lines
\newcolumntype{+}{!{\vrule width 2pt}}

% create \thickcline for thick horizontal lines of variable length
\newlength\savedwidth
\newcommand\thickcline[1]{%
	\noalign{\global\savedwidth\arrayrulewidth\global\arrayrulewidth 2pt}%
	\cline{#1}%
	\noalign{\vskip\arrayrulewidth}%
	\noalign{\global\arrayrulewidth\savedwidth}%
}

% \thickhline command for thick horizontal lines that span the table
\newcommand\thickhline{\noalign{\global\savedwidth\arrayrulewidth\global\arrayrulewidth 2pt}%
	\hline
	\noalign{\global\arrayrulewidth\savedwidth}}


% Remove comment for double spacing
%\usepackage{setspace} 
%\doublespacing

% Text layout
% \raggedright
\setlength{\parindent}{0.5cm}
\textwidth 5.25in 
\textheight 8.75in

% Bold the 'Figure #' in the caption and separate it from the title/caption with a period
% Captions will be left justified
\usepackage[aboveskip=1pt,labelfont=bf,labelsep=period,justification=raggedright,singlelinecheck=off]{caption}
\renewcommand{\figurename}{Fig}

% Use the PLoS provided BiBTeX style
%\bibliographystyle{plos2015}


% Remove brackets from numbering in List of References
\makeatletter
\renewcommand{\@biblabel}[1]{\quad#1.}
\makeatother

% define theorem and definition environments commands
\newtheorem{theorem}{Theorem}[section]
\newtheorem{definition}{Definition}[section]

% Header and Footer with logo
\usepackage{lastpage,fancyhdr,graphicx}
\usepackage{epstopdf}
%\pagestyle{myheadings}
\pagestyle{fancy}
\fancyhf{}
%\setlength{\headheight}{27.023pt}
%\lhead{\includegraphics[width=2.0in]{PLOS-submission.eps}}
\rfoot{\thepage/\pageref{LastPage}}
\renewcommand{\headrulewidth}{0pt}
\renewcommand{\footrule}{\hrule height 2pt \vspace{2mm}}
\fancyheadoffset[L]{2.25in}
% \fancyfootoffset[L]{1.25in}
\lfoot{\today}


\restylefloat{figure}


%% Include all macros below

\newcommand{\lorem}{{\bf LOREM}}
\newcommand{\ipsum}{{\bf IPSUM}}

\def\lf{\left\lfloor}   
\def\rf{\right\rfloor}

\def\ri{R_i}
\def\rj{R_j}
\def\kmi{k_{M_i}}
\def\khi{k_{H_i}}
\def\hji{H_{j_i}}
\def\ma{\overline{M}_a}
\def\ha{\overline{H}_a}
\def\mnu{M_\nu}
\def\hnu{H_\nu}
\def\myd{\text{diff}}
\def\ka{\bar{k}_\alpha}
\def\mji{M_{j_i}}

%% END MACROS SECTION
\IfFileExists{upquote.sty}{\usepackage{upquote}}{}
\begin{document}
	\vspace*{0.2in}
	
	% Title must be 250 characters or less.
	% \begin{flushleft}
	{\Large	
		\textbf\newline{Theoretical properties of nearest-neighbor distance distributions and novel metrics for high dimensional bioinformatics data} % Please use "sentence case" for title and headings (capitalize only the first word in a title (or heading), the first word in a subtitle (or subheading), and any proper nouns).	
	}
	%\newline
	% Insert author names, affiliations and corresponding author email (do not include titles, positions, or degrees).
	\begin{center}
		\begin{tabular}{l}
			Bryan A. Dawkins$^{\text{1}}$, Trang T. Le$^{\text{2}}$ and Brett A. McKinney$^{\text{1,3,}*}$ \\
			$^{\text{1}}$Department of Mathematics, University of Tulsa, Tulsa, OK 74104, USA \\
			$^{\text{2}}$Department of Biostatistics, Epidemiology and Informatics, University of \\
			\hphantom{2}Pennsylvania, Philadelphia, PA 19104 \\
			$^{\text{3}}$Tandy School of Computer Science, University of Tulsa, Tulsa, OK 74104, USA. \\
			$*$Correspondence: brett.mckinney@gmail.com
		\end{tabular}
	\end{center}
		
	% \end{flushleft}
	% Please keep the abstract below 300 words
	\section*{Abstract}
	The performance of nearest-neighbor feature selection and prediction methods depends on the metric for computing neighborhoods and the distribution properties of the underlying data. The effects of the distribution and metric, as well as the presence of correlation and interactions, are reflected in the expected moments of the distribution of pairwise distances. We derive general analytical expressions for the mean and variance of pairwise distances for $L_q$ metrics for normal and uniform random data with $p$ attributes and $m$ instances. We use extreme value theory to derive results for metrics that are normalized by the range of each attribute (max - min). In addition to these expressions for continuous data, we derive similar analytical formulas for a new metric for genetic variants (categorical data) in genome-wide association studies (GWAS). The genetic distance distributions account for minor allele frequency and transition/transversion ratio. We introduce a new metric for resting-state functional MRI data (rs-fMRI) and derive its distance properties. This metric is applicable to correlation-based predictors derived from time series data. Derivations assume independent data, but empirically we also consider the effect of correlation. These analytical results and new metrics can be used to inform the optimization of nearest neighbor methods for a broad range of studies including gene expression, GWAS, and fMRI data. The summary of distribution moments and detailed derivations provide a resource for understanding the distance properties for various metrics and data types.
	%\linenumbers

\end{document}

